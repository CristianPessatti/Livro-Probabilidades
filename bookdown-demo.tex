% Options for packages loaded elsewhere
\PassOptionsToPackage{unicode}{hyperref}
\PassOptionsToPackage{hyphens}{url}
%
\documentclass[
]{book}
\usepackage{amsmath,amssymb}
\usepackage{iftex}
\ifPDFTeX
  \usepackage[T1]{fontenc}
  \usepackage[utf8]{inputenc}
  \usepackage{textcomp} % provide euro and other symbols
\else % if luatex or xetex
  \usepackage{unicode-math} % this also loads fontspec
  \defaultfontfeatures{Scale=MatchLowercase}
  \defaultfontfeatures[\rmfamily]{Ligatures=TeX,Scale=1}
\fi
\usepackage{lmodern}
\ifPDFTeX\else
  % xetex/luatex font selection
\fi
% Use upquote if available, for straight quotes in verbatim environments
\IfFileExists{upquote.sty}{\usepackage{upquote}}{}
\IfFileExists{microtype.sty}{% use microtype if available
  \usepackage[]{microtype}
  \UseMicrotypeSet[protrusion]{basicmath} % disable protrusion for tt fonts
}{}
\makeatletter
\@ifundefined{KOMAClassName}{% if non-KOMA class
  \IfFileExists{parskip.sty}{%
    \usepackage{parskip}
  }{% else
    \setlength{\parindent}{0pt}
    \setlength{\parskip}{6pt plus 2pt minus 1pt}}
}{% if KOMA class
  \KOMAoptions{parskip=half}}
\makeatother
\usepackage{xcolor}
\usepackage{longtable,booktabs,array}
\usepackage{calc} % for calculating minipage widths
% Correct order of tables after \paragraph or \subparagraph
\usepackage{etoolbox}
\makeatletter
\patchcmd\longtable{\par}{\if@noskipsec\mbox{}\fi\par}{}{}
\makeatother
% Allow footnotes in longtable head/foot
\IfFileExists{footnotehyper.sty}{\usepackage{footnotehyper}}{\usepackage{footnote}}
\makesavenoteenv{longtable}
\usepackage{graphicx}
\makeatletter
\def\maxwidth{\ifdim\Gin@nat@width>\linewidth\linewidth\else\Gin@nat@width\fi}
\def\maxheight{\ifdim\Gin@nat@height>\textheight\textheight\else\Gin@nat@height\fi}
\makeatother
% Scale images if necessary, so that they will not overflow the page
% margins by default, and it is still possible to overwrite the defaults
% using explicit options in \includegraphics[width, height, ...]{}
\setkeys{Gin}{width=\maxwidth,height=\maxheight,keepaspectratio}
% Set default figure placement to htbp
\makeatletter
\def\fps@figure{htbp}
\makeatother
\setlength{\emergencystretch}{3em} % prevent overfull lines
\providecommand{\tightlist}{%
  \setlength{\itemsep}{0pt}\setlength{\parskip}{0pt}}
\setcounter{secnumdepth}{5}
\usepackage{booktabs}
\usepackage{amsthm}
\makeatletter
\def\thm@space@setup{%
  \thm@preskip=8pt plus 2pt minus 4pt
  \thm@postskip=\thm@preskip
}
\makeatother
\ifLuaTeX
  \usepackage{selnolig}  % disable illegal ligatures
\fi
\usepackage[]{natbib}
\bibliographystyle{apalike}
\IfFileExists{bookmark.sty}{\usepackage{bookmark}}{\usepackage{hyperref}}
\IfFileExists{xurl.sty}{\usepackage{xurl}}{} % add URL line breaks if available
\urlstyle{same}
\hypersetup{
  pdftitle={Teoria da Probabilidade},
  pdfauthor={Bruno Wavrzenczak - Cristian Pessatti dos Anjos - Caio Gomes Alves - Anderson Amorim},
  hidelinks,
  pdfcreator={LaTeX via pandoc}}

\title{Teoria da Probabilidade}
\author{Bruno Wavrzenczak - Cristian Pessatti dos Anjos - Caio Gomes Alves - Anderson Amorim}
\date{2024-01-29}

\begin{document}
\maketitle

{
\setcounter{tocdepth}{1}
\tableofcontents
}
\hypertarget{introduuxe7uxe3o}{%
\chapter{INTRODUÇÃO}\label{introduuxe7uxe3o}}

prob prob prob

\hypertarget{revisuxe3o---operadores-luxf3gicos}{%
\chapter{REVISÃO - OPERADORES LÓGICOS}\label{revisuxe3o---operadores-luxf3gicos}}

PRRRROB

\hypertarget{revisuxe3o---teoria-dos-conjuntos}{%
\chapter{REVISÃO - TEORIA DOS CONJUNTOS}\label{revisuxe3o---teoria-dos-conjuntos}}

\hypertarget{conjuntos}{%
\section{Conjuntos}\label{conjuntos}}

Chamaremos de conjunto (usualmente representado por alguma letra maiúscula) uma coleção de elementos de algum espaço maior chamado universo (representado aqui pela letra maiúscula U)

Exemplo:

\[
\begin{align*}
&U = \mathbb{R}\\
&A = \{0,2,4,6,8,10\} = \{x = 2k\;|\;k=0,1,2,3,4,5\}\\
&B = \{...,-3,-2,-1,0,1,2,3,...\} = Z\\
&C = \{x \in U\;|\;-1<x \le 2\} = (-1,2\rbrack
\end{align*}
\]

\hypertarget{intervalos}{%
\section{Intervalos}\label{intervalos}}

\[
\begin{align*}
&\lbrack a,b\rbrack = \{x\in R\;|\; a\le x\le b,\; a<b\}\\
&(a,b) = \{x \in R\;|\;a<x<b,\;a<b\}\\
\end{align*}
\]
Se \(a = b\), então temos um intervalo degenerado, também chamado de singleton
\(\;\lbrack a,a\rbrack = \{a\}\)

\hypertarget{operauxe7uxf5es-com-conjuntos}{%
\section{Operações com Conjuntos}\label{operauxe7uxf5es-com-conjuntos}}

\hypertarget{uniuxe3o-ou-conjunuxe7uxe3o}{%
\subsection{União (ou Conjunção):}\label{uniuxe3o-ou-conjunuxe7uxe3o}}

\(A \cup B = \{x \in U\;|\;x\in A\vee x\in B\}\)

Exemplo:
\[
\begin{align*}
&A=\{2,4,6,8,10\}\\
&B=\{0,1,2,3,4\}\\
&A\cup B=\{0,1,2,3,4,6,8,10\}
\end{align*}
\]

\hypertarget{interseuxe7uxe3o-ou-disjunuxe7uxe3o}{%
\subsection{Interseção (ou Disjunção):}\label{interseuxe7uxe3o-ou-disjunuxe7uxe3o}}

\(A\cap B = \{x\in U\;|\;x\in A\wedge x\in B\}\)

Exemplo:
\[
\begin{align*}
&A=\{2,4,6,8,10\}\\
&B=\{0,1,2,3,4\}\\
&A\cap B=\{2,4\}
\end{align*}
\]

\hypertarget{complementar-ou-negauxe7uxe3o}{%
\subsection{Complementar (ou Negação):}\label{complementar-ou-negauxe7uxe3o}}

\(A^{c} = \{x\in U\;|\;x\notin A\}\)

Exemplo:
\[
\begin{align*}
&U=\{1,2,3,4,5\}\\
&A=\{3,4\}\\
&A^{c}=\{1,2,5\}
\end{align*}
\]

\hypertarget{leis-de-morgan}{%
\subsection{Leis de Morgan:}\label{leis-de-morgan}}

Sejam A e B conjuntos em um universo, então vale que:

\[
\begin{align*}
&I) (A\cup B)^{c} = A^{c}\cap B^{c}\\
&II) (A\cap B)^{c} = A^{c}\cup B^{c}
\end{align*}
\]

Generalizando:

Sejam \(A_1,A_2,...,A_n\;(n\in N)\) conjuntos que pertencem a um mesmo universo, então vale que:

\$\$
\begin{align*}
&I)\;(\overset{n}{\underset{i=1}{\cup}} A_i)^{c}=
\overset{n}{\underset{i=1}{\cap}} A_i^c
\quad \text{ou}\quad(A_1\cup A_2\;...\;\cup\;A_n)^{c}=(A_1^{c}\cap A_2^{c}\;...\;\cap\;A_n^c)\\

&II)\;(\overset{n}{\underset{i=1}{\cap}} A_i)^{c}=
\overset{n}{\underset{i=1}{\cup}} A_i^{c}
\quad \text{ou}\quad(A_1\cap A_2\;...\;\cap\;A_n)^{c}=(A_1^{c}\cup A_2^{c}\;...\;\cup \;A_n^c)\\
\end{align*}
\$\$

obs: também funciona para infinitos conjuntos.

Demonstração: (Magalhães, pág.3)

Queremos demonstrar que \((\overset{n}{\underset{i=1}{\cup}} A_i)^{c}=\overset{n}{\underset{i=1}{\cap}} A_i^c\). Quando lidamos com igualdade entre conjuntos, mostraremos que cada um dos conjuntos está contido no outro, portanto precisamos verificar duas condições:

\((\overset{n}{\underset{i=1}{\cup}} A_i)^{c}\subset\overset{n}{\underset{i=1}{\cap}} A_i^c\quad\)e\(\quad(\overset{n}{\underset{i=1}{\cup}} A_i)^{c}\supset\overset{n}{\underset{i=1}{\cap}} A_i^c\)

Seja \(\omega\) um elemento qualquer pertencente ao universo, então supomos que \(\omega\in(\overset{n}{\underset{i=1}{\cup}} A_i)^{c}\), portanto \(\omega\notin(\overset{n}{\underset{i=1}{\cup}} A_i)\), o que também leva a conclusão de que \(\omega\notin A_i\) para todo \(i\). Logo \(\omega\in A_i^{c}\) para todo \(i\) e, portanto \(\omega\in\overset{n}{\underset{i=1}{\cap}} A_i^c\)

De forma análoga, podemos facilmente verificar a segunda condição e, portanto, demonstramos a primeira lei de Morgan. A demonstração da segunda lei pode ser feita da mesma forma.

\hypertarget{exercuxedcios-resolvidos}{%
\section{Exercícios Resolvidos}\label{exercuxedcios-resolvidos}}

\textbf{(Magalhães, seção 1.2) 1.} Sendo A, B e C subconjuntos quaisquer, expresse em notação matemática os conjuntos cujos elementos:

\textbf{a.} Estão em A e B, mas não em C.

\textbf{b.} Não estão em nenhum deles.

\textbf{c.} Estão, no máximo, em dois deles.

\textbf{d.} Estão em A, mas no máximo em um dos outros.

\textbf{e.} Estão na intersecção dos três conjuntos e no complementar de A.

\textbf{Soluções:}

\textbf{a.} não estar em \(C\) significa estar no complementar de \(C\), portanto podemos reescrever a questão da seguinte forma: ``Estão em \(A\) e \(B\) e no complementar de \(C\)''. Quando dizemos que um elemento está em um conjunto \textbf{E} em outro, estamos falando de uma interseção, portanto a solução é:

\(A\cap B\cap C^{c}\)

\textbf{b.} não estar em nenhum deles é a negação de estar em todos eles. Logo a solução é a seguinte:

\((A\cup B\cup C)^{c}\) ou, utilizando a primeira lei de Morgan: \(A^{c}\cap B^{c}\cap C^{c}\)

\textbf{c.} o elemento não pode estar nos três conjuntos, o que é a negação de estar nos três conjuntos:

\((A\cap B\cap C)^{c}\)

\textbf{d.} isso significa estar em \(A\) e não estar na interseção dos outros dois conjuntos:

\(A\cap(B\cap C)^c\)

\textbf{e.} para estar nos três conjuntos precisa também estar em \(A\), o que torna a interseção com \(A^c\) um conjunto vazio \((\phi)\)

\hypertarget{experimentos-e-eventos-probabiluxedsticos}{%
\chapter{EXPERIMENTOS E EVENTOS PROBABILÍSTICOS}\label{experimentos-e-eventos-probabiluxedsticos}}

aqui a esperança morre

\hypertarget{probabilidade}{%
\chapter{PROBABILIDADE}\label{probabilidade}}

prob prob prob prob prob

\hypertarget{probabilidade-condicional}{%
\chapter{PROBABILIDADE CONDICIONAL}\label{probabilidade-condicional}}

prob prob prob prob prob prob

\hypertarget{variuxe1veis-aleatuxf3rias}{%
\chapter{VARIÁVEIS ALEATÓRIAS}\label{variuxe1veis-aleatuxf3rias}}

blalblbalblablal PROOOOOB

\hypertarget{esperanuxe7a-e-variuxe2ncia}{%
\chapter{ESPERANÇA E VARIÂNCIA}\label{esperanuxe7a-e-variuxe2ncia}}

OIASDVAHPVIUSDVADNFIPABDFI

\hypertarget{funuxe7uxe3o-geradora-de-momentos-e-funuxe7uxe3o-caracteruxedstica}{%
\chapter{FUNÇÃO GERADORA DE MOMENTOS E FUNÇÃO CARACTERÍSTICA}\label{funuxe7uxe3o-geradora-de-momentos-e-funuxe7uxe3o-caracteruxedstica}}

oadusvovoo

\hypertarget{modelos-de-probabilidade}{%
\chapter{MODELOS DE PROBABILIDADE}\label{modelos-de-probabilidade}}

probprobporbbbbbbbbbbbbbbbbbbbbbbbbbbbbbbs

\hypertarget{funuxe7uxf5es-de-variuxe1veis-aleatuxf3rias}{%
\chapter{FUNÇÕES DE VARIÁVEIS ALEATÓRIAS}\label{funuxe7uxf5es-de-variuxe1veis-aleatuxf3rias}}

PROOOOOOOOOOOOOOOOOOOOOOOOOOOOOOOOOOOOOOOOOOOOOOOOOOOOOOOOOOOOOOOOOOOOOOOOOOOOOOOOOOOOOOOOOOOOOOOOOOOOOOOOOOOOOOOOOOOOOOOOOOOOOOOOOOOOOOOOOOOOOOOOOOOOOOOOOOOOOOOOOOOOOOOOOOOOOOOOOOOOOOOOOOOOOOOOOOOOOOOOOOOOOOOOOOOOOOOOOOOOOOOOOOOOOOOOOOOOOOOOOOOOOOOOOOOOOOOOOOOOOOOOOOOOOOOOOOOOOOOOOOOOOOOOOOOOOOOOOOOOOOOOOOOOOOOOOOOOOOOOOOOOOOOOOOOOOOOOOOOOOOOOOOOOOOOOOOOOOOOOOOOOOOOOOOOOOOOOOOOOOOOOOOOOOOOOOOOOOOOOOOOOOOOOOOOOOOOOOOOOOOOOOOOOOOOOOOOOOOOOOOOOOOOOOOOOOOOOOOOOOOOOOOOOOOOOOOOOOOOOOOOOOOOOOOOOOOOOOOOOOOOOOOOOOOOOOOOOOOOOOOOOOOOOOOOOOOOOOOOOB

  \bibliography{book.bib,packages.bib}

\end{document}
