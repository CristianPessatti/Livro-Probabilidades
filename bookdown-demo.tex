% Options for packages loaded elsewhere
\PassOptionsToPackage{unicode}{hyperref}
\PassOptionsToPackage{hyphens}{url}
%
\documentclass[
]{book}
\usepackage{amsmath,amssymb}
\usepackage{iftex}
\ifPDFTeX
  \usepackage[T1]{fontenc}
  \usepackage[utf8]{inputenc}
  \usepackage{textcomp} % provide euro and other symbols
\else % if luatex or xetex
  \usepackage{unicode-math} % this also loads fontspec
  \defaultfontfeatures{Scale=MatchLowercase}
  \defaultfontfeatures[\rmfamily]{Ligatures=TeX,Scale=1}
\fi
\usepackage{lmodern}
\ifPDFTeX\else
  % xetex/luatex font selection
\fi
% Use upquote if available, for straight quotes in verbatim environments
\IfFileExists{upquote.sty}{\usepackage{upquote}}{}
\IfFileExists{microtype.sty}{% use microtype if available
  \usepackage[]{microtype}
  \UseMicrotypeSet[protrusion]{basicmath} % disable protrusion for tt fonts
}{}
\makeatletter
\@ifundefined{KOMAClassName}{% if non-KOMA class
  \IfFileExists{parskip.sty}{%
    \usepackage{parskip}
  }{% else
    \setlength{\parindent}{0pt}
    \setlength{\parskip}{6pt plus 2pt minus 1pt}}
}{% if KOMA class
  \KOMAoptions{parskip=half}}
\makeatother
\usepackage{xcolor}
\usepackage{longtable,booktabs,array}
\usepackage{calc} % for calculating minipage widths
% Correct order of tables after \paragraph or \subparagraph
\usepackage{etoolbox}
\makeatletter
\patchcmd\longtable{\par}{\if@noskipsec\mbox{}\fi\par}{}{}
\makeatother
% Allow footnotes in longtable head/foot
\IfFileExists{footnotehyper.sty}{\usepackage{footnotehyper}}{\usepackage{footnote}}
\makesavenoteenv{longtable}
\usepackage{graphicx}
\makeatletter
\def\maxwidth{\ifdim\Gin@nat@width>\linewidth\linewidth\else\Gin@nat@width\fi}
\def\maxheight{\ifdim\Gin@nat@height>\textheight\textheight\else\Gin@nat@height\fi}
\makeatother
% Scale images if necessary, so that they will not overflow the page
% margins by default, and it is still possible to overwrite the defaults
% using explicit options in \includegraphics[width, height, ...]{}
\setkeys{Gin}{width=\maxwidth,height=\maxheight,keepaspectratio}
% Set default figure placement to htbp
\makeatletter
\def\fps@figure{htbp}
\makeatother
\setlength{\emergencystretch}{3em} % prevent overfull lines
\providecommand{\tightlist}{%
  \setlength{\itemsep}{0pt}\setlength{\parskip}{0pt}}
\setcounter{secnumdepth}{5}
\usepackage{booktabs}
\usepackage{amsthm}
\makeatletter
\def\thm@space@setup{%
  \thm@preskip=8pt plus 2pt minus 4pt
  \thm@postskip=\thm@preskip
}
\makeatother
\ifLuaTeX
  \usepackage{selnolig}  % disable illegal ligatures
\fi
\usepackage[]{natbib}
\bibliographystyle{apalike}
\IfFileExists{bookmark.sty}{\usepackage{bookmark}}{\usepackage{hyperref}}
\IfFileExists{xurl.sty}{\usepackage{xurl}}{} % add URL line breaks if available
\urlstyle{same}
\hypersetup{
  pdftitle={Teoria da Probabilidade},
  pdfauthor={Bruno Wravzenczak - Cristian Pessatti dos Anjos},
  hidelinks,
  pdfcreator={LaTeX via pandoc}}

\title{Teoria da Probabilidade}
\author{Bruno Wravzenczak - Cristian Pessatti dos Anjos}
\date{2024-01-24}

\begin{document}
\maketitle

{
\setcounter{tocdepth}{1}
\tableofcontents
}
\hypertarget{introduuxe7uxe3o}{%
\chapter{INTRODUÇÃO}\label{introduuxe7uxe3o}}

prob prob prob

\hypertarget{revisuxe3o---operadores-luxf3gicos}{%
\chapter{REVISÃO - OPERADORES LÓGICOS}\label{revisuxe3o---operadores-luxf3gicos}}

PRRRROB

\hypertarget{revisuxe3o---teoria-dos-conjuntos}{%
\chapter{REVISÃO - TEORIA DOS CONJUNTOS}\label{revisuxe3o---teoria-dos-conjuntos}}

Chamaremos de conjunto (usualmente representado por alguma letra maiúscula) uma coleção de elementos de algum espaço maior chamado universo (representado aqui pela letra maiúscula U)

\hypertarget{exemplo}{%
\subsection{Exemplo:}\label{exemplo}}

\[
\begin{align}
U = R\\
A = \{0,2,4,6,8,10\} = \{x = 2k\;|\;k=0,1,2,3,4,5\}\\
B = \{...,-3,-2,-1,0,1,2,3,...\} = Z\\
C = \{x \in U\;|\;-1<x \le 2\} = (-1,2\rbrack
\end{align}
\]

\hypertarget{experimentos-e-eventos-probabiluxedsticos}{%
\chapter{EXPERIMENTOS E EVENTOS PROBABILÍSTICOS}\label{experimentos-e-eventos-probabiluxedsticos}}

aqui a esperança morre

\hypertarget{probabilidade}{%
\chapter{PROBABILIDADE}\label{probabilidade}}

prob prob prob prob prob

\hypertarget{probabilidade-condicional}{%
\chapter{PROBABILIDADE CONDICIONAL}\label{probabilidade-condicional}}

prob prob prob prob prob prob

\hypertarget{variuxe1veis-aleatuxf3rias}{%
\chapter{VARIÁVEIS ALEATÓRIAS}\label{variuxe1veis-aleatuxf3rias}}

blalblbalblablal PROOOOOB

\hypertarget{esperanuxe7a-e-variuxe2ncia}{%
\chapter{ESPERANÇA E VARIÂNCIA}\label{esperanuxe7a-e-variuxe2ncia}}

OIASDVAHPVIUSDVADNFIPABDFI

\hypertarget{funuxe7uxe3o-geradora-de-momentos-e-funuxe7uxe3o-caracteruxedstica}{%
\chapter{FUNÇÃO GERADORA DE MOMENTOS E FUNÇÃO CARACTERÍSTICA}\label{funuxe7uxe3o-geradora-de-momentos-e-funuxe7uxe3o-caracteruxedstica}}

oadusvovoo

\hypertarget{modelos-de-probabilidade}{%
\chapter{MODELOS DE PROBABILIDADE}\label{modelos-de-probabilidade}}

probprobporbbbbbbbbbbbbbbbbbbbbbbbbbbbbbbs

\hypertarget{funuxe7uxf5es-de-variuxe1veis-aleatuxf3rias}{%
\chapter{FUNÇÕES DE VARIÁVEIS ALEATÓRIAS}\label{funuxe7uxf5es-de-variuxe1veis-aleatuxf3rias}}

PROOOOOOOOOOOOOOOOOOOOOOOOOOOOOOOOOOOOOOOOOOOOOOOOOOOOOOOOOOOOOOOOOOOOOOOOOOOOOOOOOOOOOOOOOOOOOOOOOOOOOOOOOOOOOOOOOOOOOOOOOOOOOOOOOOOOOOOOOOOOOOOOOOOOOOOOOOOOOOOOOOOOOOOOOOOOOOOOOOOOOOOOOOOOOOOOOOOOOOOOOOOOOOOOOOOOOOOOOOOOOOOOOOOOOOOOOOOOOOOOOOOOOOOOOOOOOOOOOOOOOOOOOOOOOOOOOOOOOOOOOOOOOOOOOOOOOOOOOOOOOOOOOOOOOOOOOOOOOOOOOOOOOOOOOOOOOOOOOOOOOOOOOOOOOOOOOOOOOOOOOOOOOOOOOOOOOOOOOOOOOOOOOOOOOOOOOOOOOOOOOOOOOOOOOOOOOOOOOOOOOOOOOOOOOOOOOOOOOOOOOOOOOOOOOOOOOOOOOOOOOOOOOOOOOOOOOOOOOOOOOOOOOOOOOOOOOOOOOOOOOOOOOOOOOOOOOOOOOOOOOOOOOOOOOOOOOOOOOOOOB

  \bibliography{book.bib,packages.bib}

\end{document}
